\documentclass[letterpaper,10pt,titlepage,draftclsnofoot,onecolumn,utf8,latin1]{IEEEtran}
\usepackage{graphicx}
\usepackage{amssymb}
\usepackage{amsmath}
\usepackage{array}
\usepackage{amsthm}
\usepackage{listings}
\usepackage{alltt}
\usepackage{float}
\usepackage{color}
\usepackage{url}
\usepackage{setspace}
\usepackage{balance}
\usepackage[TABBOTCAP, tight]{subfigure}
\usepackage{enumitem}
\usepackage{pstricks, pst-node}
\usepackage[utf8]{inputenc}
\usepackage[margin=.75in]{geometry}
\usepackage{titlesec}
\usepackage{fancyhdr}
\usepackage{hyperref}
\usepackage{geometry}
\usepackage{tocloft}

%hide toc subsubsections
\setcounter{tocdepth}{2}
\setlength{\parindent}{.5in}

%toc formatting for IEEE 830-1998 standards
\renewcommand{\cftsecleader}{\cftdotfill{\cftdotsep}{\vspace{.25cm}}}
\renewcommand{\cftsecfont}{\normalfont}
\renewcommand{\cftsecpagefont}{\normalfont}
\renewcommand{\cftsecaftersnum}{.}

%bottom right page numbers
\fancyhf{}
\renewcommand{\headrulewidth}{0pt}
\rfoot{\thepage}
\pagestyle{fancy}

%formatting specific IEEE 830-1998 Section headings
\titleformat{\section}[block]
  {\fontsize{11}{10}\bfseries\sffamily}
  {\thesection.}
  {1em}
  {\vspace{.1cm}}
\titleformat{\subsection}[block]
  {\fontsize{10}{10}\slshape\sffamily}
  {\thesubsection}
  {1em}
  {\vspace{.1cm}}
\titleformat{\subsubsection}[block]
  {\fontsize{12}{10}\slshape\sffamily}
  {\thesubsubsection}
  {1em}
  {\vspace{.1cm}}
  
\geometry{textheight=8.5in, textwidth=6in}

\renewcommand{\thesection}{\arabic{section}}
\renewcommand{\thesubsection}{\thesection.\arabic{subsection}}

\newcommand{\cred}[1]{{\color{red}#1}}
\newcommand{\cblue}[1]{{\color{blue}#1}}

\def\name{Charles Siebert, Branden Berlin, Yipeng "Roger" Song}

%% The following metadata will show up in the PDF properties
\hypersetup{
  citecolor = black,
  urlcolor = black,
  pdfauthor = {\name},
  pdfkeywords = {cs461 ``Technology Review'' writing},
  pdftitle = {CS461 - Technology Review},
  pdfsubject = {CS461 Writing},
  pdfpagemode = UseNone
}

\title{CS461-Writing Project}

\author{\name}

\begin{document}

\begin{titlepage}
\centering
\vspace*{9cm}
{\scshape\LARGE Technology Review and Implementation } \\
	{\scshape\Large CS461 - Fall 2016 \par}
	\vspace{.5cm}
	\name \par
    {\large \today \par} 
	\vspace*{1cm}
    
\begin{abstract}
The technology of Virtual Reality (VR) currently is not cost effective to today's market, as the cost of high-end setups required makes it difficult to afford. Browser developers are focusing primarily on expensive high-end high-performance hardware over mobile devices for Augmented Reality (AR) or Virtual Reality (VR) on the web. Doing AR/VR on the mobile web allows more developers to enter the field and deliver to more customers. To accomplish this, we are working on a project called �Mobile AR/VR Performance�, which focuses on researching to profile and identify performance bottlenecks in 3D web content on mobile devices. We will file issues in the open source projects for Chrome, Firefox through A-Frame and Three.js to determine and identify those bottlenecks. We hope to accomplish this by reporting the challenges and opportunities for performance VR/AR applications, and write a blog post detailing the project results and their best-practices.
\end{abstract}
\end{titlepage}

\newpage

\thispagestyle{empty}
\pagenumbering{gobble}

\begin{singlespace}
\tableofcontents

\cleardoublepage
\pagenumbering{arabic}

\newpage

\section{Introduction}
Our project "Optimizing Virtual Reality and Augmented Reality Performance on Mobile Web Applications," nicknamed "OVRAR" is the development of a VR application and analyzing the performance based on the restrictions of using it in a mobile environment with the same devices. Using the data we collect from the application, we can determine specific bottlenecks and performance issues that may be incurred on the device based on different test implementations. This document outlines the overview of technologies that will be used during the lifeline of the project. Each technology listed will have three tools that could be used to accomplish the specific task, and a comparison between them that will lead to a conclusion of which tool was chosen, and why. The purpose of this document is to break down each section of the project that is required to determine how we will implement the program and how the data will be analyzed.

Every three sections will be authored by each of us, leaving us 9 sections of technologies to be listed. The first three (Section 1 through Section 3) is written by [Name] the second three (Section 4 through Section 6) is written by [Name], and the last three (Section 7 through Section 9) is written by [Name]. Each person authoring their sections will ultimately be responsible for those technologies they discuss and conclude to use to accomplish the tasks of the system and the capturing of data.

\section{The Technologies}

\section{Creation of Scenes in A-frame (Roger)}
\subsection{Options to Use}
As the purpose of this project is to determine areas of development within A-Frame where practices will be best used, so we are going to use A-Frame to create scenes.
A-Frame is an open-source WebVR framework for creating virtual reality (VR) experiences with
HTML with the use of the Three.js framework
\subsection{Evaluation of Options}
\subsection{Options Comparison}
Other options are HTMLs, or JavaScripts. (need to expand)
\subsection{Discussion}
\subsection{Selection}

\section{Generation Of Scenes In Browser}
\subsection{Options to Use}
Each of the web browsers are capable software applications for retreiving, presenting, and traversing information resources that are not only identified by a URL but an IP address as well and are capable of hosting a web page, image, video, or other piece of content such as 3D imaging.  
\subsection{Evaluation of Options}
Each of these options are potential web browsers that are capable to loading local servers via Apache. 
\subsection{Options Comparison}
While each of these options are testable web browsers, they may require different coding aspects and each option might require different libraries or coding tactics (i.e. Explorer does not like certain HTML and CSS variable phrases and structures that chrome and firefox do) and would require extra time and effort to transition code.
\subsection{Discussion}
Though we may be able to test in each of these browsers, our client has asked us to use Mozilla Firefox because the is the browsers environment that our research project is based around and that is the company our client works for.
\subsection{Selection}
Mozilla Firefox

\section{Generation and capture of the data}
\subsection{Options to Use}
There's a couple of options to use, specifically we're looking at the Android SDK, which has a bunch of good features that are included with the capturing of data through different levels of the android stack, where there's performance metrics with the phone, the operating system, and within the browser. A competing tool is the Firefox Dev tools, which allows the connection of devices over a network, and will capture performance metrics through the device onto the browser. This holds information such as performance usage, memory consumption and, and rendering speeds of the scene.
\subsection{Evaluation of Options}
Android SDK, Firefox Dev Tools, Chrome Dev Tools.
\subsection{Options Comparison}
\subsection{Discussion}
These tools will all actually be used simultaneously, as they are recommended by our client to test multiple tools as we conduct our research. More tools allow us to have a wider set of test cases to determine possible areas of optimization issues and bottlenecks.
\subsection{Selection}
All of them are to be used, sections need some expanding here.

\section{Processing/Analyzing Data}
\subsection{Options to Use}
Options for the recording and analyzing of data are: Excel, Google Sheets, Apache OpenOffice, LibreOffice, and many more.
\subsection{Evaluation of Options}
Each of these tools are used for analyzing and recording data and some over others are better at manipulating data to either find patterns of present trends. 
\subsection{Options Comparison}
Though each of these options are capable of recording general data, some of them offer more features then others and some are more complicated than this project's data recording needs to be. 
\subsection{Discussion}
For our project, we are justing going to be recording software and hardware specs. This means we don't need to use fancy data processing tools and we don't need to be plotting points of greating graphs. We just need a simple data tool and Excel is readily available and reliable.
\subsection{Selection}
Excel

\section{Generation of scenes on Mobile Device}
\subsection{Options to use}
Our client requires us to use a specific mobile device specifically for the sake of having the same device to have a similar testing case for our research. The purpose of the Nexus 5X is to develop on a newer phone that works on an Android platform, as the project is developed on the A-Frame framework, and implemented through the Firefox and Chrome browsers. iOS and Safari are excluded from this testing.
\subsection{Evaluation of Options}
The only option is the Nexus 5X, as this is what our client requires us to use for testing development.
\subsection{Options Comparison}
\subsection{Discussion}
To be expanded.
\subsection{Selection}

\section{Setup Local Host Server}
\subsection{Options to Use}
The options for setting up locahost servers are: Nginx, Apache, and Lighttpd.
\subsection{Evaluation of Options}
These are the three most used free web-servers provided by the opensource comminuty that will allow of to remotely connect via mobile web-browser through IP address. 
\subsection{Options Comparison}
Nginx, the second most popular web hosting server software supports most OS but uses event driver architecture instead of threads. Lighttpd is the third most used server software but serves less than 1% of domains and is also event-driven. Apache is the most used server software with it being used on 60% of domains and it supports a multitude of packages. Additionally it is easier to install and provides for a much richer and reliable environment. 
\subsection{Discussion}
Because we don't know the situations we are going to run into through our web hosting, we are going to want as server that is not only more reliable and easier to use, but we are potentially going to want to apply different packages, and because apache holes the majority of user use, we can trust that it will be able to cater to our needs and have more up-to-date features. 
\subsection{Selection}
Apache 

\section{Development Tools (Roger)}
\subsection{Options to Use}
In this project, we are going to use Visual Studio as the development tool. Visual Studio is often used to develop computer programs for Microsoft Windows, as well as web sites, web applications and web services. 
\subsection{Evaluation of Options}
\subsection{Options Comparison}
Compare visual studio with Note++, Sublime and Eclipse.
\subsection{Discussion}
\subsection{Selection}

\section{Language Tools or Framework (Roger)}
\subsection{Options to Use}
A implementation language is a formal computer language or constructed language designed to communicate
instructions to a machine, particularly a computer. i.e. HTML, JavaScript, C, C++, etc. For our project, since we are going to test the performance of web app through A-frame, which is a Web framework, so we decided to use HTML and Java Scripts to implement the test scripts. 
\subsection{Evaluation of Options}
\subsection{Options Comparison}
(May compare the language we choose with python, c, or c++)
\subsection{Discussion}
\subsection{Selection}

\section{Version Control and Documentation}
\subsection{Options to Use}
There's multiple forms of version control, Github being the main one, and LaTeX being our word processing document for easy of use within our choice of version control.
\subsection{Evaluation of Options}
The options to choose from in terms of version control range from Github, to Dropbox, and Google Docs. Github, being the widely used one, will house the project implementation and documentation of our results from our project.
\subsection{Options Comparison}
\subsection{Discussion}
To be expanded.
\subsection{Selection}

\section{Conclusion}

\end{singlespace}

\begin{thebibliography}{9}

\bibitem{ref1} 
Authors of Ref1
\textit{Ref1 Title}. 
Ref1 Publisher

\end{thebibliography}

\end{document}
