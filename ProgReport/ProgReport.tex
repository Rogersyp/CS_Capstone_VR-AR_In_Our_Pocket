\documentclass[letterpaper,10pt,draftclsnofoot,onecolumn]{IEEEtran}
\usepackage{graphicx}
\usepackage{amssymb}
\usepackage{amsmath}
\usepackage{array}
\usepackage{amsthm}
\usepackage{listings}
\usepackage{alltt}
\usepackage{float}
\usepackage{color}
\usepackage{url}
\usepackage{setspace}
\usepackage{balance}
\usepackage{enumitem}
\usepackage{pstricks, pst-node}
\usepackage{inputenc}
\usepackage[margin=.75in]{geometry}

%crappy error with titlesec
\newcommand{\subparagraph}{}
\usepackage{titlesec}

\usepackage{fancyhdr}
\usepackage{hyperref}
\usepackage{tocloft}

%hide toc subsubsections
\setcounter{tocdepth}{2}
\setlength{\parindent}{.25in}

%toc formatting for IEEE 830-1998 standards
\renewcommand{\cftsecleader}{\cftdotfill{\cftdotsep}{\vspace{.25cm}}}
\renewcommand{\cftsecfont}{\normalfont}
\renewcommand{\cftsecpagefont}{\normalfont}
\renewcommand{\cftsecaftersnum}{.}

%bottom right page numbers
\fancyhf{}
\renewcommand{\headrulewidth}{0pt}
\rfoot{\thepage}
\pagestyle{fancy}

%formatting specific IEEE 830-1998 Section headings
\titleformat{\section}[block]
  {\fontsize{12}{12}\bfseries\sffamily}
  {\thesection.}
  {1em}
  {\vspace{.1cm}}
\titleformat{\subsection}[block]
  {\fontsize{12}{15}\slshape\sffamily}
  {\thesubsection}
  {1em}
  {\vspace{.1cm}}
\titleformat{\subsubsection}[block]
  {\fontsize{11}{20}\slshape\sffamily}
  {\thesubsubsection}
  {1em}
  {\vspace{.1cm}}
  
\geometry{textheight=8.5in, textwidth=6in}

\renewcommand{\thesection}{\arabic{section}}
\renewcommand{\thesubsection}{\thesection.\arabic{subsection}}
\renewcommand{\thesubsubsection}{\thesubsection.\arabic{subsubsection}}

\newcommand{\cred}[1]{{\color{red}#1}}
\newcommand{\cblue}[1]{{\color{blue}#1}}

\def\name{Charles Siebert, Branden Berlin, Yipeng "Roger" Song}

%% The following metadata will show up in the PDF properties
\hypersetup{
  urlcolor = black,
  pdfauthor = {\name},
  pdfkeywords = {cs461 ``Senior Capstone - Fall 2016'' capstone},
  pdftitle = {CS 461 Progress Report Fall 2016},
  pdfsubject = {Capstone Progress Report Fall 2016},
  pdfpagemode = UseNone
}

\begin{document}
\begin{titlepage}
\centering
\vspace*{6cm}
{\scshape\LARGE \begin{singlespace}Optimizing Virtual Reality and Augmented Reality Performance on Mobile Web Applications \\ \end{singlespace} \smallskip Group 52 - Progress Report } \\
	{\scshape\Large CS461 - Fall 2016 \par}
	\vspace{.5cm}
	\name \par
    {\large \today \par} 
	\vspace*{1cm}
	
\begin{abstract}
The technology of Virtual Reality (VR) currently is not cost effective to today's market, as the cost of high-end setups required makes it difficult to afford. Browser developers are focusing primarily on expensive high-end high-performance hardware over mobile devices for Augmented Reality (AR) or Virtual Reality (VR) on the web. Doing AR/VR on the mobile web allows more developers to enter the field and deliver to more customers. To accomplish this, we are working on a project called �Mobile AR/VR Performance�, which focuses on researching to profile and identify performance bottlenecks in 3D web content on mobile devices. We will file issues in the open source projects for Chrome, Firefox through A-Frame and Three.js to determine and identify those bottlenecks. We hope to accomplish this by reporting the challenges and opportunities for performance VR/AR applications, and write a blog post detailing the project results and their best-practices.
\end{abstract}

\end{titlepage}

\newpage

\thispagestyle{empty}
\pagenumbering{gobble}

\tableofcontents

\cleardoublepage
\pagenumbering{arabic}

\newpage

\begin{singlespace}

\section{Introduction}
This paper is a progress report for group 52, "OVRAR" over the past 8 weeks for the Fall term of 2016.

\section{Purpose and Goals}
The purpose of this project is to determine areas of development within A-Frame where practices will be best used, as they will least be likely to impede on bottle necking either the software or hardware when optimizing the software for performance. This project is focused towards the advancement of an open-source, developing web framework, and the developers making their own products with A-Frame and for mobile devices. The developers will be using our project research as a means to avoid these bottlenecks in this evolving environment. Developers other than us will use the information in the report to determine the best way to approach at designing their programs, as the software we create will only serve as test cases and stress testing for mobile devices to collect this information.

Optimizing VR and AR for Mobile Web Apps is to determine Virtual Reality (VR) and Augmented Reality (AR) bottlenecks that exist in mobile devices within the A-Frame framework. The bottlenecks can be caused from either unoptimized development of software, underpowered or unoptimized hardware found in existing devices, or potential bugs or limitations found within the framework itself. The software itself, which is developed on A-Frame, will generate multiple scenes where it will test the graphical capabilities of the hardware within the mobile devices, the types of different implementations of certain scenes, and determine areas of optimization through these multiple scenes. The software will be used to create a report that will analyze the information collected about processing power, frame rates, battery usage, and the limitations of the framework to determine the best practices for implementing more graphically intensive programs on A-Frame.

\section{Group Reports}
Over the past 10 weeks, each of us has submitted weekly updates to describe the progress over the past week, the problems we encountered during the week, and the plans we have for the next week. This is to keep track of our progress as we move through the project time-line, and determine whether we are on track, or behind our current schedule. Based on our time-line we have created (which is found within our Design Document), we are currently on track to finishing all the required documents needed for the term, and researching and designing test implementations over the break. This all prepares us as a group for Winter Term, and the start of implementation and testing. Included in each person's summary of the term includes a retrospective, which outlines positive things we encountered in our group, changes that need to be implemented, and steps required to implement the changes within our group.

\subsection{Charles Siebert}
The following sections include weekly summaries and a retrospective reflection over the past 10 weeks.

\subsubsection{Weekly Summaries}

Our group picks up this project at the start of week three of this term, since the projects were assigned the week before. We met for the first time this week to have our first meeting with our client to discuss the overall scope of the project. During this week, we had to make a Problem Statement that presented the project as a solution, or adding to a solution to a problem. This project specifically is about researching and optimizing an open-source framework that renders virtual and augmented reality scenes onto mobile devices. During this week, we didn't run into any problems in writing the Problem Statement, and our client was pretty clear in the purpose and scope of the project. Our challenges for the next week is understanding what needs to go into a Requirements Document and start thinking about how to approach implementing A-Frame onto mobile devices. Our client discusses with us about testing development on standardized equipment that they use, to reduce discrepancies, so he is sending us Nexus 5X phones.

The next coming week, in week four, we plan on setting up the phones we received to be used as tools for this development process. We talked after out meeting with our TA about learning the tools to use and get set up with the Hello World project in A-Frame. Since the last week, we've received feedback on our Problem Statements to revise it once more, and received the requirements for producing a Requirements Document for this project, that will be uploaded to this project's repository by October 28th. When Roger and I contacted our client in the weekly meeting, we discussed the goals of the coming week and will reconvene next Monday about our progress. We haven't had any problems this week, as being in contact with our client and group members hasn't been an issue.

Starting at week five, we are going to be finishing up our revisions to the final version of our Requirements Document. After we revised the Problem Statement and re-submitted with the proper signatures, we received an extension towards this next writing assignment. We didn't meet up with our TA at the end of the week, due to other circumstances, though we did meet with our client to talk about the upcoming work we are producing and what the document entails. During the week, we started putting together our Requirements Document, which seemed to pose a problem due to missing a session of class to properly evaluate the kind of content that goes into it. Missing a time for our group to meet made it hard for us to communicate and properly plot out a Gantt chart to describe how the coming months will play out. Our next meeting with our client is on Tuesday to go over the progress we've made and talk about how the coming week looks like.

week 6 goes here

During week 7, I worked on the revisions of the Requirements Document with our client, discussing some issues that were unclear, or heeding his advice on the length of certain sprints that we have planned out in our time-line for our project. Though we tried to have revisions and a final version turned in on Friday the previous week, it's good that we were able to discuss some pertinent details and concerns that our client had. Aside from this, we worked on discussing the parts of the Technology Review that we have to put together, all in which we have 3 parts that we are responsible for.

Our plan for the next week is finishing the Technology Review, which is due on Monday, and start the planning process for our Design Document. We have a meeting with our client on Monday to discuss in detail some of the technologies we need to utilize based on the constraints of our specific project. Overall, the biggest challenge we faced was finishing the revisions for the Requirements Document while simultaneously attempting to piece together the things we need to write for the tech review. The added time to send the document back for clarification and confirmation on the unclear sections will add some time to the process.

In week 8, we were unable to properly finish the Technology Review in time on Monday. We received an extension on the assignment so we can finish the assignment and turned it in on Wednesday at noon through the GitHub repository. One of the classes this week was cancelled, which gave us some time to look through the sections of the Design Document, and determine which parts of the system we need to describe in detail. During the cancelled class time, we were able to commit a bare bones document to start off with the Design Document to our GitHub repository. We talked with our client and discussed some technologies that their other teams have used to track the performance metrics through web-browsers on phones, which led us to some good information to include and discuss in the documents.

This coming week, we have to prepare for the last few assignments for this term, of which include the Design Document, Progress Report (Term 1) and the poster, which are all pretty time consuming. Currently our focus is getting the first revision of the Design Document out as soon as possible to get revisions and feedback as early as we can to avoid the situation we ran into with the Requirements Document (time delays between revisions caused some issues). Though this is Turkey Week next week, I will have some time to work on this during the times I get tired of my family. We didn't run into any pressing issues this week, but I do think we will have some trouble finding time during dead week and finals week to finish all of the stuff to end this term. It's essential we get some of the pressing matters finished as soon as possible.

Since week 9 has a holiday, we had little time to meet up and coordinate. It was discussed that we would make headway on the documents we need to prepare and work on our progress report that's due at the end of the term. The past week I worked on a bit of my part of the presentation and discussed how we will piece together the progress report and video, and it looks like we'll be using OBS (Open Broadcast Software) as our tool, since I've had previous experience with it. The next week, we need to have a finished version of our Design Document by Sunday at the latest, for our client to go over in time for it to be turned in on Wednesday of next week (Dec. 7th). Our challenges we face is getting enough time to meet together and get these documents all pieced together, since we have two papers, a 30 minute presentation, and a poster we need to fulfill within the span of eight days.

In the last week of the term, we had to work to get the Design Document done in time, and have it polished enough to send off to our client. We finished the Design Document, and turned it in without our client's signature as we are still waiting for a response for some feedback or revisions. The design document itself lists out the concerns of the project, where we need to determine how to overcome these concerns based on our design decisions we make. This included the tools we are using, methodology of implementing and re-implementing, and ideas of how we are going to stress the resources we have in order to determine areas of optimization and bottlenecks.

Since it is the end of the term, this is a harder deadline. We planned out what to do for the rest of the term, in terms of the poster and progress report. Essentially we are meeting up during the weekend to record our presentation for our project, where we will make and use our sections of PowerPoint slides to talk about. The challenges we face for the coming week is to find enough time to be able to write all the slides, record enough interesting content, and then still be able to put together a mock-poster and the final progress report. Though we do have a challenge over the winter break, and that's to not get distracted from the project and stay on course. We plan on doing research on the bugs currently found within A-Frame and Firefox specifically, and start our initial, simple implementations of A-Frame onto our mobile devices. This will help us in understanding the work flow that is required for this project over the coming months.

\subsubsection{Retrospective Reflection}

\begin{center}
    \begin{tabular}{ | p{0.3\linewidth} | p{0.3\linewidth} | p{0.3\linewidth} | }
    \hline
    \bfseries Positives & \bfseries Deltas & \bfseries Changes Necessary \\ \hline
    Positive & Delta & and some changes \\ \hline
    Positive & Delta & and some changes \\ \hline
    Positive & Delta & and some changes \\ \hline    
    \end{tabular}
\end{center}

\subsection{Branden Berlin}

\subsubsection{Weekly Summaries}
Stuff here. \bigskip

\subsubsection{Retrospective Reflection}

\begin{center}
    \begin{tabular}{ | p{0.3\linewidth} | p{0.3\linewidth} | p{0.3\linewidth} | }
    \hline
    \bfseries Positives & \bfseries Deltas & \bfseries Changes Necessary \\ \hline
    Positive & Delta & and some changes \\ \hline
    Positive & Delta & and some changes \\ \hline
    Positive & Delta & and some changes \\ \hline    
    \end{tabular}
\end{center}

\subsection{Yipeng "Roger" Song}

\subsubsection{Weekly Summaries}
Stuff Here.

\subsubsection{Retrospective Reflection}

\begin{center}
    \begin{tabular}{ | p{0.3\linewidth} | p{0.3\linewidth} | p{0.3\linewidth} | }
    \hline
    \bfseries Positives & \bfseries Deltas & \bfseries Changes Necessary \\ \hline
    Positive & Delta & and some changes \\ \hline
    Positive & Delta & and some changes \\ \hline
    Positive & Delta & and some changes \\ \hline    
    \end{tabular}
\end{center}

\end{singlespace}

\end{document}
