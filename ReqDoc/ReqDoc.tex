\documentclass[letterpaper,10pt,titlepage,draftclsnofoot,onecolumn,compsoc,utf8,latin1]{IEEEtran}
\usepackage{graphicx}
\usepackage{amssymb}
\usepackage{amsmath}
\usepackage{array}
\usepackage{amsthm}
\usepackage{listings}
\usepackage{alltt}
\usepackage{float}
\usepackage{color}
\usepackage{url}
\usepackage{setspace}
\usepackage{balance}
\usepackage[TABBOTCAP, tight]{subfigure}
\usepackage{enumitem}
\usepackage{pstricks, pst-node}
\usepackage[utf8]{inputenc}
\usepackage[margin=.75in]{geometry}
\usepackage{titlesec}
\titleformat{\section}[block]
  {\fontsize{12}{15}\bfseries\sffamily}
  {\thesection}
  {1em}
  {}
\geometry{textheight=8.5in, textwidth=6in}

\newcommand{\cred}[1]{{\color{red}#1}}
\newcommand{\cblue}[1]{{\color{blue}#1}}

\usepackage{hyperref}
\usepackage{geometry}

\def\name{Charles Siebert, Branden Berlin, Yipeng "Roger" Song}

%% The following metadata will show up in the PDF properties
\hypersetup{
  urlcolor = black,
  pdfauthor = {\name},
  pdfkeywords = {cs461 ``Senior Capstone - Fall 2016'' capstone},
  pdftitle = {CS 461 Requirements Document},
  pdfsubject = {Capstone Requirements Document},
  pdfpagemode = UseNone
}

\begin{document}
\begin{titlepage}
\centering
\vspace*{4cm}
{\scshape\LARGE [Project Name] Requirements } \\
	{\scshape\Large CS461 - Fall 2016 \par}
	\vspace{.5cm}
	\name \par
    {\large \today \par} 
    
	\vspace*{1cm}
	
\begin{abstract}
Abstract goes here
\end{abstract}

\end{titlepage}

\tableofcontents
\newpage
\section{Overview}
\noindent
This is the text for the overview.

\subsection{Scope}
\noindent
This is the text for the scope.

\section{Glossary}
\begin{enumerate}
    {\item \bfseries Web Framework: } Something
    {\item \bfseries A-Frame: } Something
    {\item \bfseries Implementation Languages: } Something
    {\item \bfseries Mobile Devices: } Something
    {\item \bfseries Rendering: } Something
    {\item \bfseries Bottleneck: } Something
    {\item \bfseries Optimize: } Something
    {\item \bfseries Performance: } Something
    {\item \bfseries [word]: } Something
\end{enumerate}

\section{The "What" of the Project}
\begin{singlespace}
Functionality - The software is supposed to be runnable in a mobile environment to determine bottlenecks between the software and the hardware.\\

\noindent
External Interface - The software will determine how well it works between the hardware.\\

\noindent
Performace - The speed is dependant on the type of hardware used.
Availability is large due to the immense market that the mobile devices provide. The software has to be responsive to human interactions, which would include tools to respond to user input, and changing the view of the scene based on the view-port and position the phone provides.\\


\noindent
Attributes - In terms of portability and maintainability, HTML markups and reducing boiler plate functions allows for transferring projects easily and seamlessly.
Correctness will be based off of how well we can push the boundaries of our software, based on the performance the hardware can provide.\\

\noindent
Security - N/A, as this is a proof of concept, this does not lie within our scope.\\

\noindent
Constraints - The biggest constraint is implementing on mobile devices, as these end up typically being weaker than normal hardware that is used for similar, tough, unoptimized graphics rendering that desktop computers are able to provide.
\end{singlespace}

\section{Gantt Chart}



    

\end{document}