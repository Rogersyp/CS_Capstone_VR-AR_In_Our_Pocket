\documentclass[letterpaper,10pt,titlepage,draftclsnofoot,onecolumn,compsoc,utf8,latin1]{IEEEtran}
\usepackage{graphicx}
\usepackage{amssymb}
\usepackage{amsmath}
\usepackage{array}
\usepackage{amsthm}
\usepackage{listings}
\usepackage{alltt}
\usepackage{float}
\usepackage{color}
\usepackage{url}
\usepackage{setspace}
\usepackage{balance}
\usepackage[TABBOTCAP, tight]{subfigure}
\usepackage{enumitem}
\usepackage{pstricks, pst-node}
\usepackage[utf8]{inputenc}
\usepackage[margin=.75in]{geometry}
\usepackage{titlesec}
\titleformat{\section}[block]
  {\fontsize{12}{15}\bfseries\sffamily}
  {\thesection}
  {1em}
  {}
\geometry{textheight=8.5in, textwidth=6in}

\newcommand{\cred}[1]{{\color{red}#1}}
\newcommand{\cblue}[1]{{\color{blue}#1}}

\usepackage{hyperref}
\usepackage{geometry}

\def\name{Charles Siebert, Branden Berlin, Yipeng "Roger" Song}

%% The following metadata will show up in the PDF properties
\hypersetup{
  urlcolor = black,
  pdfauthor = {\name},
  pdfkeywords = {cs461 ``Senior Capstone - Fall 2016'' capstone},
  pdftitle = {CS 461 Requirements Document},
  pdfsubject = {Capstone Requirements Document},
  pdfpagemode = UseNone
}

\begin{document}
\begin{titlepage}
\centering
\vspace*{4cm}
{\scshape\LARGE [Project Name] Requirements } \\
	{\scshape\Large CS461 - Fall 2016 \par}
	\vspace{.5cm}
	\name \par
    {\large \today \par} 
    
	\vspace*{1cm}
	
\begin{abstract}
The technology of Virtual Reality (VR) currently is not cost effective to the today's market, as the cost of high-end setups required makes it difficult to afford. Browser developers are focusing primarily on high-end specialized hardware for VR on mobile, and are generally ignoring Augmented Reality (AR). Therefore, doing AR on the web allows far more developers to enter the field. To accomplish this, we are working on a project called “Mobile Web App”, which focuses on profiling and identifying performance bottlenecks in 3D web content and camera acquisition/usage on mobile devices. We will file issues in the open source projects for Chrome, Firefox, A-Frame and Three.js to determine and identify those bottlenecks. We hope to accomplish this by reporting the challenges and opportunities for performant VR/AR applications, and write a blog post detailing the project results and their best-practices.
\end{abstract}

\end{titlepage}

\tableofcontents
\newpage
\section{Overview}
\noindent
This is the text for the overview.

\subsection{Scope}
\noindent
This is the text for the scope.

\section{Glossary}
\begin{enumerate}
    {\item \bfseries Web Framework: } A-Frame
    {\item \bfseries A-Frame: } A web framework for building virtual reality experiences
    {\item \bfseries Implementation Languages: } We will be using Java for Android-based programming
    {\item \bfseries Mobile Devices: } Nexus
    {\item \bfseries Rendering: } Android User Interfeace
    {\item \bfseries Bottleneck: } Testing the software against the hardware.
    {\item \bfseries Optimize: } Android Performance Profiling
    {\item \bfseries Performance: } Based on Nexus capabilities
    {\item \bfseries [word]: } N/A
\end{enumerate}

\section{The "What" of the Project}
\begin{singlespace}
Functionality - The software will be utilized in an Android environment and tested against various web apps such as Chrome and Firefox.\\

\noindent
External Interface - The external interface will be command-line programming in Windows/Linux.\\

\noindent
Performace - The performanceof the softawre against the hardware is what is going to be tested in this project.
Availability is large due to the immense market that the mobile devices provide. The software has to be responsive to human interactions, which would include tools to respond to user input, and changing the view of the scene based on the view-port and position the phone provides.\\


\noindent
Attributes - In terms of portability and maintainability, HTML markups and reducing boiler plate functions allows for transferring projects easily and seamlessly.
Correctness will be based off of how well we can push the boundaries of our software, based on the performance the hardware can provide.\\

\noindent
Security - N/A, as this is a proof of concept, this necessity does not align within our scope.\\

\noindent
Constraints - The biggest constraint is implementing on mobile devices, as these end up typically being weaker than normal hardware that is used for similar, tough, unoptimized graphics rendering that desktop computers are able to provide.
\end{singlespace}

\section{Gantt Chart}



    

\end{document}